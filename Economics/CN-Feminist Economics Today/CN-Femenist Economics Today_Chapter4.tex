\documentclass[UTF8]{ctexart}
\title{女性主义理论及种族经济不平等}
\author{Lisa Saunders, William Darity Jr.}

\begin{document}
\maketitle


\textit{“在人类学习的过程中,个人或群体灌输的概念、意义、习惯、或观点将决定其理解或评估数据,沟通意见,和规范行为的方式。”}


\textit{Vernon Dixon, 《The Di-unital Approach to Economics: A Black Perspective》}




对于种族主义,传统经济学理论们将种族的不平等待遇归因于前市场事件,外生性偏好或信息不对等。这些理论却不足以解释在经济机会和经济状态的种族不平等。他们当中促成了学界对数据的误读的概念,习惯与意义,约束了学者之间意见的交流,影响学者们在分析政策时的行为,以及他们对与自己不同群体的态度。
种族不平等和歧视的持续存在,表明存在着建构种族规范与界限的物质激励。为了更好地理解种族规范,和它们随时间和经济状况的持续存在,对这种激励的调查是必要的。这种方法的重点在于识别那些发起种族界限的建构,以建立和维持种族特权(通常是经济方面),如此来分给自己财富,社会地位,和对“他人”控制权的人。当种族界限被打破时,考虑到权力的所有权,通常是优势群体选择打破的条件与位置。比如南方奴隶制中的性侵犯,导致了“混血”(mulatto)儿童的出现,他们往往是男性白人奴隶主性侵女性黑人奴隶的产物。这一章探讨通过加入女性主义理论的视角,来加强关于种族与种族不平等的经济理论的进一步发展(1)。
在本书的前言部分已经讨论过,女性主义理论告诉我们“性”(sex)与“性别”(gender)是明显不同的概念。“性”指人类男性或女性的生理身份。而“性别”,则指与文化中依惯例被指定为“男性化的”或“女性化的”实践相关的行为。如此,在一些社群中,一个做家务的男人(生理意义上)并不是一个“男性”(文化意义上)。而一个参与接触运动的女人也并非一个“女性”。因此,“性”在很大程度上是固定的,同时“性别”则更加具有流动性。歧视可以基于根植在生理上性别间的显著区别,也可以基于不同于生理性别的文化预期。
这种女性主义理论所论述的,“性”与“性别”的显著区别指出了身份聚焦的经济理论与种族意识的经济理论之间的鸿沟,与种族和种族主义的区别。种族是一种社会分类,通过外表,血统或二者兼具的方式将人类分成不同的群体。而种族主义则是对这种分类的调动,以建立群体间的优劣,从而分别给予高等与低等群体随之而来的特权与束缚。
种族,不同于性,人类对于群体间差异的标志存在一定的武断性,从而造成了其虚幻的性质。比如肤色,发质,发色,出身等等。当今,体质人类学家们几近完全放弃了基于DNA中心特征确定种族类别的项目。但种族主义绝对不是虚构,它在相当广大的范围内被实践:在日本,对部落民(Buraku)的打压;在卢旺达,布隆迪和前南斯拉夫的种族仇杀;在北爱尔兰,新教徒凌驾于天主教徒之上;在印度,贱民(Dalits)被拥有更高种姓的印度教徒贬低等等。
因此,我们需要一种“种族意识的”经济理论来全面理解我们的世界。而种族意识,则是为群体不平等与不公正开出良药的必需品。我们若是要解决种族主义的影响,就不能忽视种族本身。所以,我们的目标并非制定一种种族中立的,“色盲的”分析框架,而是制定一种允许我们达到一个种族公平社会的分析框架。
女性主义理论指出,相较于基于性的歧视(sex-based discrimination),基于种族的歧视与基于性别的歧视(gender-based discrimination)具有更多的相似性,因为种族身份的根基同性别一样根植于文化当中。女性主义观点必定增进我们调查和解决种族不平等的能力吗?其方法论在多大程度上不同于传统经济学?其会向我们暴露更多的,更广的,没有被以传统视角审视世界的经济学家所考虑到的问题吗?具体来说,我们解决了这样一个问题,即这样的方法是否能够阐明社会构建的种族分化产生不平等的过程。

\section{关于种族主义及不平等的新古典理论}当新古典经济学家理论化基于种族的不平等时,他们传统地,以劳工市场运作的缺陷为背景做出解释(2)。在雇佣和薪酬方面截然不同的待遇被臆断为源自前市场事件,外生性偏好(未能赢得劳工市场中的竞争),雇主对潜在雇员的品质缺乏了解,或三者兼具。
包括劳工市场在内,机构制度中特定的相互作用增减不平等程度的方式,及许多其他关于公平程度的考量往往没有得到调查,而社会分裂的历史起源与制度源头也同样如此。在美国这些起源包括了奴隶制、南部的吉姆克劳(Jim Crow)实践、劣等学校、和对美国黑人积累财产的能力的剥夺。希望从经济学的有利角度中讨论这类问题的的学者们经常被建议在另一门社会科学追寻它们的答案。(3)
经济学因此被定义为一个将关于拍卖和拍卖市场的无休止的理论极大地特权化,以至凌驾于种族与性别议题之上的学科空间。大多数经济学家将对种族不平等的全面研究视作超出界限,或对经济研究的正常流程的负面干扰。我们则指出对种族差距与歧视的进一步综合考量是经济学中非常重要的议题,而将其囊括在领域的研究中心是非常值得的努力。
经济学家所受的训练使他们不成比例地过分关注经济行为的效率,而非公平。而如果一个经济学家欣然采纳帕累托原理(Pareto Principle),他/她便无法排列出最有效率的分配。理性的行为者,当得知具体的成本和收益时,则会被自利驱动。个人动机及其约束的差异有时可被传统模型所适应,而其他了解和选择因素的方式则被忽视,尤其是当它们与模型中的传统行为假设冲突时。构建模型要求对许多现实细节的抽象化,而我们希望加以考量的,则是抽象的特定选择如何削弱用于解释社会分裂和群体间不平等的理论效力。
新古典经济学家通常会假设:市场中的利益最大化参与者,在生产率没有实质性差异的情况下,不会持续纵容导致差异待遇和薪酬的偏见行为。因此,在劳工市场中的歧视是不可持续的,且在经济产出中可见的,可衡量的落差则取决于先于市场运作的因素。如此,对于某个特定种族的工人来说,更低的薪酬说明,相较于更高薪酬的工人,他们一定具有平均生产率上的不足,以此解释这样的落差。现在相当热门的故事之一便是,黑人与白人在劳工市场产出上的落差是因为黑人的认知能力不足。(4)
此外,在新古典的范式中,种族偏好的源头通常并不被研究,而是被当作一种既定事实——就像一个人更喜欢番茄汤而非洋葱汤一样。社会风俗,习惯和信仰形成的方式因此被忽视,然而它们本身的存在和它们对行为的影响被设想为是在决定产出时的关键因素。(5)
比如,弗农·迪克森(1971)认为个人主义,作为深嵌于美国文化基质的价值观,渗透在经济学科当中。而随着个人主义被作为一种默认准则,新古典主义者普遍认为每个人的效用函数(utillity function)具有独立性和自主性。这导致人们无法看清社会的独立性与互动性产生的行动,换句话说,就是学习成为“白人”或“黑人”的含义,学习自己在社会中的“位置”。事实上,标准经济学很大程度上将认识,实践与决策排除于理论活动之外。(6)
对于区别对待同样资格的申请者的决策者,新古典主义者认为在信息条件昂贵的条件下,或更笼统地说,不完美的市场下,这样是理性的。他们认为在信息昂贵时,雇主根据群体的平均值来做出是否雇佣个人的决策是合理的。市场中的雇主是不完美的,比如一个单一雇主市场,他/她可能出于理性地在雇佣和薪酬决策上考虑种族偏好。同样,只要有利可图,雇主在决策中顾及白人(或男性)客户和雇员的赤裸偏见就被认为是合理的。
如前文所述,许多经济学家认为,因为对求职者的生产率差异的计量和控制不足,导致了劳工市场的歧视被夸大其词。通过将种族和性别歧视视作理性抉择或某种市场失灵的结果,新古典经济学建构了一种叙事,使歧视行为变得更难以察觉,更无足轻重,最终更有理可循。的确,在这样的角度下,种族主义实践便不基于种族主义。
所以,许多群体的代理存在超过了经济学传统实践的范畴,除少数情况下,经济学家将群体合理地还原成群体成员们的个人。群体的社会性存在,在品味和偏好被视作外生性的,或仅仅被视作个人的个人性和非社会性发展。
Paula England(1993,本书第一章)描述了一种分离的自我,其体现在个人主义,与外生、不变的、主导着古典经济学的偏好,还有一种相关的对人际间幸福感与满足感的比较的长期禁锢之中。她认为,正是这种在新古典经济学中的禁锢,使得对不平等与弱势的分析被置于经济思想的边缘地带。

\section{种族规范的起源及作用} 种族规范的构建提供了产生压迫的机制。(7)然而,经济行为的个体的准确动机是多维度的。在某些情况下中种族与性别歧视以不同的方式表现,在另一些情况下则相似,但它们必然要共同作用。这二者在决定结果时的影响力视具体情况而定。
比如,当一个雇主、地主、信贷经理(loan officer),或其他情况类似的人,做出个人决策时,他/她会根据他/她认为自己所了解的有关个人的一切来形成印象。 充分证据表明,这样的决策很可能不仅取决于个人的正式资格,而是也取决于决策者对个人所归属的群体的看法即他/她对该群体的认识。因此,最终决定(关于雇佣,薪酬,晋升,解雇,住房,信贷)则基于正式和非正式的评估。
定性研究往往能让我们深入了解某些雇主信奉及奉行的种族规范,与工人们的反应。例如 Browne 和 Kennelly(1999) 研究了雇主对于不平等待遇是否因“封闭”因素而起的看法。“封闭”这一概念在社会学中用于描述旨在维护自身所属群体利益的个人或群体行为。通过比较在亚特兰大,佐治亚有雇员概况的地区中雇主在采访中的回应,研究者们发现白人男性雇主在回应关于他们的工人的开放式问题的时候相当同义,而且他们常常是不正确的,并对他们的工人抱有成见。
这些雇主高估了家中有年幼子女的女性的比例,也高估了黑人女性在单身母亲中的比例。他们也高估了女性为照料儿童所损失的时间,并低估了男性在同样情况下中损失的时间。相较于黑人女性,他们在谈论白人女性的时候要更加正面,更具有同情心(8),而自愿谈论黑人男性的雇主们在此时则更加负面,也更缺乏同情心。

\section{理解种族与性别的互动}上述的雇主回应说明,他们对女性及任何导致性别歧视的态度被种族化(racialized),而同时,他们的种族态度和基于种族的成见都被性别化(gendered)了。仅本项研究中,雇主的言论就充分表明了将女性主义分析融入种族歧视相关研究,并将种族分析融入性别歧视相关研究的重要性。(9)
原则上,我似乎可以直截了当地认为白人女性必然只会受到性别歧视,而黑人女性则要同时面对种族歧视和性别歧视。所以,黑人女性所面对的歧视水平应当要更高些,且可以通过减去性别歧视对白人女性收入的影响,来剔除纯粹的种族歧视对黑人女性收入的影响。
然而,这种将种族和性别层面的歧视影响直接相加的假设,很可能并不成立。此外,也有实证异常的存在。大多数旨在检验美国劳动力市场中歧视的统计工作,尤其是针对黑人女性与白人女性的歧视,并没有发现相关证据。(10)
但无论如何,断定黑人女性并没有面对种族歧视是错误的。与上述统计研究形成鲜明对比的,是对歧视存在与否的直接测试。在这样的审计研究中,受过培训的演员们将被提供相同的简历,并被指导进行面试。黑人和白人演员们随后申请同样的工作,并记录面试者的行动及考量他们在应聘过程中的进展情况。这些测试中研究者往往发现,在招聘的各个阶段,与白人女性相比,黑人女性都更受到严重歧视:她们不太可能受邀参与面试、得到工作以及得知未被宣布的晋升机会,而且,即使得到工作,她们也很可能得到更低的工资(Bendick,Jackson, 和 Reinoso, 1997)。我们如何调和这些相互矛盾的发现?
统计研究通常对工人的就业部门进行控制,有时以相当宽大的层面上,有时则以较为细致的方面上,一边最终将不同就业类别的薪酬差异视为既定事实。这些差异对于黑人男性来说仍然相当严重,而对于女性,无论肤色,则更加严重。Conard(2001)指出即使在二十世纪九十年代,仍能发现受过大学教育的黑人女性集中在低薪酬的就业部门中。这可能是将女性,无论肤色地分配到性别分类的工作,而将其中的黑人女性分配到性别分类和种族分类的工作当中,随之而来的则是更低的薪酬待遇(Simms 和 Malveaux, 1986)。在这种情况下,为控制就业类别所进行的工资回归可能发现女性当中的种族歧视。Hamilton 和 Darity(2002)发现有证据表明,这种拥挤导致了男性中的黑人与白人的薪酬差异。在其研究脚注中他们解释道,黑人女性与白人女性的对比揭示了黑人女性获得的工作类型存在的差异。然而,这种工作差异并不与黑人女性和白人女性的就业薪酬差异相关。
Hamilton 独立地意识到,在控制动机(个人电子邮件沟通)上的失误导致了未能找到对黑人女性薪酬上的歧视。他的这一结论,来自于 Goldsmith, Veum 和 Darity(2000b 及其它)的大量研究,这些研究使用了全国青年纵向研究(National longitudinal Study of Youth, NLSY)的数据,当将动机作为因变量时,他们发现了黑人妇女在薪酬方面受到歧视的证据。相比于拥有相似计量人力资本的白人女性,家庭与社区义务的差异可能给予黑人女性更大的工作动机。当审计研究派出动机匹配的申请人时,我们便会期待其能发现雇主的歧视行为。不控制动机的种族差异的统计分析,从另一方面来说,很可能会混淆家庭压力与歧视的残差,从而低估歧视。若是没有对不同,但又相互作用的种族影响与性别影响对经济结果之作用的敏锐认识,我们便无法看清这些歧视在模式上的差异,自然也不会考虑如何解释这些差异。我们也不会知道,可能需要针对反歧视措施加以调整,以适应黑人男性和黑人女性在不同情况下的低薪待遇。
女性的分歧在于她们在劳工市场内的互动在多大程度上基于肤色或是性别。Higginbotham 和 Weber(1999)研究了专业管理工作者对他们的劳动经历的态度及反应。他们采访了孟菲斯(Memphis)地区的女性工作者以寻找黑人女性和白人女性对自己在工作中所受待遇的异同。两个群体都同样强调了歧视的群体经历,却忽视了个人经历。同时,它们也都同样意识到了晋升障碍,并列举出了某些相似的障碍,“相比于白人女性,黑人女性在工作场所经历了更多以及更多样的歧视待遇”(350)。研究中,黑人女性表示她们常常观察到待遇上的种族差异,而同时,多数白人女性则表示他们从未观察到对任何种族女性的种族歧视。
基于种族和性别的排斥性可能在某些方面表现相同,而在其他方面不同;而且,他们也可能独立表现或一同表现取决于其在研究中的具体互动关系。然而新古典劳动经济学家们仍对不同形式的压迫采用一样的方法。这表明,他们仍设想雇主们对所有种族群体采用一样的排斥策略,并不分种族(或阶级,性取向等等)地在性别排斥中采用这些相同的策略。与此相反地,雇主实际上很可能会根据他们对每个群体和情况的特有评估,或偏见来做出决策。因此,在不同的种族群体、时间、经济条件等方面,排斥往往以不同的方式进行。(11)无疑,在这项位于孟菲斯的调查中,女性感知到,并可能经历过类似或不同类型的排斥。对于种族和性别接纳与排斥人们的方式中存在的巨大差异,了解这些差异本身,它们交互和它们影响结果的方式是非常必要的。同时,如上述所言,以非此即彼的方式理论化种族和性别会限制经济分析的进行。
直至最近,许多对种族就业和薪酬不平等的公开研究仍然专注于黑人男性,而像排除其他种族群体的男性和女性一样排除黑人女性。类似地,对女性的经济研究直至最近仍然主要研究白人女性的就业,薪酬和家庭经理。
劳工市场的确根据种族和性别将工人接纳或排斥在职业,这对某些群体来说比其他群体更加不利。群体劣势在不同地区和时间的实际形式,其严重程度也不尽相同。(12)所以,我们当然可以理解,从这样一个单一重点中可以了解许多针对特定群体的信息。然而我们认为,其并不能了解关于种族或性别不平等的充足信息。当把某个群体单独分出,结果会显示某个类别(种族、性别等等)对经济机会产生了最大的影响,而将其它方面的歧视则会被忽视,压制,或被置于次要。
Higginbotham 和 Weber(1999)描述了,黑人女性观察到的种族问题,并将其视为群体间合作的阻碍,白人女性在观察并承认这些问题时的失误。他们认为这种承认的缺失导致了“对她们的黑人同事的潜在阻碍”,并减少了“白人女性和黑人同事联合起来反对工作场所中的种族不公正的潜力”。他们的结论是,“只有通过比较研究,记录种族对白人女性和黑人女性的意义,方能评估这种联合的基础”(349-350)。比较研究,对于正在研究种族不平等的性别本质与性别不平等的种族本质的经济学家是相当有用的。(13)

\section{理论化女性主义和反种族主义}我们能从 Higginbotham 和 Weber 学到的一个重点是,在抛弃了种族、性别、或任何一个压迫的必要基础的单独性的学者们之间,他们建立强有力的联合的可能性。尽管女权主义和反种族主义的学术研究道路并非同路,但仍有许多共同点。这两个学界的研究者同样都与将他们的理论归类为“边缘”和“叛逆”的霸权作斗争,同样都采取多样化的研究方法,从不同于主流的角度概念化理论,并使用传统与非传统的方法论,并为政策贡献新的见解。
许多经济学家都寻求理解根据种族身份区别对待人们的经济动机与结果。认为不平等与不公正待遇是长期建立的社会规范的逻辑结果的学者们,研究这类规范的发展和维持,虽然许多这种研究都是在经济学以外的学科范畴中完成的,(14)许多经济学家也仍在寻求对种族规范的理解。(15)
为了利于全体女性,对经济机会不平等的女性主义研究需要包含种族分析。(16)不幸的是,这并没有在多数的女性主义经济学研究中实现。当 Dixon 写道,“白人发现,在认识黑人的同时否认他们,是可能跟黑人共鸣的”(1971,30),他解释道,对于那些以黑白思维看得黑人的白人来说,与黑人的交往能力是一种矛盾,他们认为黑人不是白人,是另类。他接着说道,他们通过让所有白人的东西具有普遍性与权威性来解决这一矛盾(同见 Charusheela和 Zein-Elabdin,本书第八章)。研究有色人中生活的女性主义之间可能存在这样的冲突。对抗性别歧视的观念是反对压迫其他群体(或至少是压迫其他群体中的女性)的重要部分,这一观将广大女性主义者,尤其是有色人种女性主义者带入了两难境地。白人女性主义者可以轻易地与有色人种女性以女性身份共鸣,而通过忽视种族在她们遭遇中所扮演的角色,并将所有反性别歧视的东西变得具有普遍性和权威性来“同时否认她们”。如果关于性别剥削的女性主义模型仅仅基于特定白人女性的经历,其对解决有色人种女性的不利处境作用就相当有限,而我们认为这样是遗憾的,因为女性主义可以为反种族主义经济学提供信息与帮助。
同理,反种族主义学者们难以忽视女性主义。即使不是大多数,也有许多有色人种女性与他人的互动是至少由性别决定的。包容性的,关于种族不平等的反种族主义经济理论因此会变得女性主义化,并且认同对有色人种的歧视的性别本质。对种族不平等充分的女性主义分析,是性别的,经济不平等的种族化分析,而且需要复杂的分析工具,这些工具中的部分在新古典经济学范式之外才更为常见。这样的理论要考虑到,个体在群体内和群体间的价值观和生活方式都各不相同。其需要考虑到自私和无私,经济和非经济的动机。并且将制度性种族主义视为一种排他,可以协调一致的群体行为,或具有多层面动机的持续而广泛的个人行为的间接结果。同时,如前所述,于种族和性别的压迫性可能在某些方面表现相同,而在其他方面不同,并且一同出现。它们互动并随着时间影响行为和经济条件的方式是种族化的女性主义分析的必要组成。
而随着新兴研究展示了对种族主义对经济结果的影响理解的良好迹象,其似乎有些零散。(迄今为止,仍没有相似倾向的学者做出集中努力。)而我们认为,其他学科所做的工作并没有提供足够的信息。但是,许多有趣的问题已经引起了我们的主意,而许多可敬的学者们也正在寻找答案。在这些问题中,有相当数量的问题都很基本,但也很复杂:种族规范来源于何处?它们如何改变?他们一直是性别针对性的吗?在什么样的经济和制度条件下种族规范才会被加强或削弱?什么类型的政策可以被用来增加种族规范的成本,包括偏见?其它类型的压迫(恐同,宗教偏见等等)如何与种族主义交互?

\section{结论}在《超越经济人》中,Williams 认同应当“种族化我们的(女性主义)性别理论”,因为毕竟,在后殖民文化中建立起来的经济思想,即科学家宣布种族是对人类的合法区分(1993,144-45)。我们同意她的观点,即很早就应该进行种族化的女性主义分析,而十年后的今天,这一要求仍然迫切;并且我们为此提出了更多的理由。
在这章,我们论述了我们认为经济理论的基本结构边缘化种族不平等的方式,以及经验性的研究策略如何淡化,甚至有时无法发现种族歧视的表现。我们认为女性主义和反种族主义学者们可以互惠互利,尽可能承认并应用这样的基本原则:不同种族群体内和群体间的经济结果存在性别差异,不同性别群体内和群体间的经济结果而存在种族差异。我们利用定量和定性研究的结果来证明这一点。我们建议经济学家们报告他们的发现并接纳更多复杂的方法和思想,包括那些在来自于其他学科的,无论如何这都有助于他们研究极不平等的性别和种族本质。

\section{注释}
(1)  美国的社会分化往往基于以下一个或多个方面:年龄、肤色、残疾、性别、种族、宗教、国籍、性取向和体型。我们认为,在分析美国的社会分化时,仅考虑种族和性别是不全面的,应该全面考虑个人。然而,在本章中,我们充分关注了对种族歧视以及在一定程度上对性别歧视的经济分析。


(2)雇用和薪酬以外的不平等现象虽然相当严重,但却很少受到关注。当前人口调查数据显示,2000 年西班牙裔和黑人的周收入中位数分别只有白人的 67\%和 80\%。毫无疑问,劳动力市场收入不平等是一个需要重点分析的问题。然而,其他类型的严重而持久的不平等却很少受到经济学家的关注。Chiteji 和 Stafford(1999)引用的研究结果表明,根据数据来源和 "家庭财富 "的定义,黑人/白人的平均财富比率从 0.17 到 0.25 不等,中位财富比率从 0.08 到 0.13不等。人口普查 1999 年的报告显示,26.1\%的黑人和 25.9\%的西班牙裔人生活在贫困线以下,而白人的这一比例为8.2\%。此外,根据城市研究所 1999 年的一份报告(见 Fix and Struyk 1993;Ackerman et al.)。


(3)  关于试图将制度和历史纳入经济分析的另一个论述,见本书第七章,Blank 和 Reimers。


(4)提出这一观点的标准参考文献是 Neal 和 Johnson,1996。有几项研究对这一论断提出了质疑,但往往被认知技能缺陷立场的支持者所忽视。其中之一是罗杰斯和斯普里格斯 1996 年的研究,该研究探讨了不同种族在认知技能方面产生可测量差异的过程中的种族差异。另一项研究是 Goldsmith、Veum 和 Darity(2000a)的著作,该著作表明,即使考虑到 Neal 和 Johnson 所使用的相同的认知技能测量方法,在同时考虑心理资本的情况下,仍然存在很强的歧视性差异。


(5) 关于偏好形成的进一步讨论,见 England,本书第一章;以及 Blank 和 Reimers,本书第七章。


(6) 也许除了一小撮与圣达菲(Santa Fe)研究所有联系的特立独行的数学经济学家,他们明确关注邻近效应和传染效应。


(7) 当从属群体将强加的身份认同作为一种骄傲的工具并加以修改时,身份认同就会成为一种反抗的工具。


(8) 另一项研究引述了妻子或女儿从事专业工作的白人男性管理人员的同情评论(Zetlin,1999)。作者采访的一些高管表示,当他们了解到妻子或女儿所面临的不利处境时,他们自己对女性管理者的态度和行为都有所改善。其中一些男性实际上发起了支持女性的计划。目前还不清楚这种行为上的改变是否或在多大程度上延伸到了非白人女性经理人身上。


(9) Evelyn Nakano Glenn(1985)曾就此撰文。


(10) 统计工作中普遍发现黑人妇女不受歧视,但 Marlene Kim(2002 年)最近利用 2000 年当前人口调查(CPS)数据所做的研究是个例外。


(11) 例如,见 Darity 和 Deshpande,2000 年,关于几个不同国家种族不平等的最新变化的讨论,以及 Darity,1989 年,其中举例说明了美国移民之间经济机会的差异,即使他们来自同一种族群体。另见 England, Christopher, and Reid,1999 和 Cintrón-Velez,1999。(12) 有关美国妇女的多种族经济史,见 Amott 和 Matthaei,1996。


(12) 例如,1998 年参加平权行动的 Reskin。


(13) 尤其见 Browne 和 Kennelly,1999;Higginbotham 和 Weber,1999 年;以及 Reskin,1998。


(14) 见 Darity,1989;Williams,1993;Matthaei,1996;Darity、Mason 和 Stewart,2000;King 和 Easton,2000;以及 Spriggs 和 Williams,2000。


(15) 为了我们的目的,我们将女性主义学术定义为一种致力于消除女性不平等待遇的认识论。当然,女性主义还有其他定义;有关概述,见 Nelson,1993。


(16) 见 Williams,1993 和 Matthaei,1996。



\end{document}