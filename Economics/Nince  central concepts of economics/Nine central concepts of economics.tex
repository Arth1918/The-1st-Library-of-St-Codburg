\documentclass[UTF8]{ctexart}

\title{经济学九大中心概念}
\author{Arthur Li}
\date{8/4/2024}

\begin{document}
\maketitle
\begin{enumerate}
    \item 稀缺性:相对于社会的无限需求下,经济资源的有限。
    \item 决策:稀缺性导致需求不能被完全满足,因此需要决策以分配资源。
    \item 效率:有效产出与总输入的比率
    \item 平等:经济产出被分配给经济体中不同群体,分配的相似性。
    \item 经济福祉:一个多维概念,指当前和未来的财务安全、满足基本需求的能力与长期保持适当的收入水平。
    \item 可持续性:在不牺牲未来需求满足能力的情况下,满足现代需求的能力。
    \item 变革:在机构、结构、科技、经济或社会层面的改变。对于理解经济不断变动的状态非常重要。
    \item 相互依存:在经济体中,各经济因素的相互影响。更高的相互依存度意味着更广泛的影响。在对高度相互依存的世界进行经济分析时必须考虑此因素。
    \item 干预:市场/经济体运行中的政府参与。
\end{enumerate}

\end{document}