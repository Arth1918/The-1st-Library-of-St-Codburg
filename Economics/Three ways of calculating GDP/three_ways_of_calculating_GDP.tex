\documentclass{article}

\title{Three ways of calculating GDP}
\author{Arthur}
\date{2024 Oct. 17th}

\begin{document}
\maketitle

\section{The Output Method}
This method measures the goods and services produced within the economy.

\begin{equation}
GDP = GVA + T + S
\end{equation}

$GVA$ - Gross Value Added, the goods and services produced by all the sectors\footnote{Primary sector, Secondary sector and Tertiary sector, usually not including Quaternary sector and Quinary sector} within the economy.

$T$ - Taxes

$S$ - Subsidies

\section{The Income Method}
This method measures the incomes earned in the economy, the GDP calculated with method is referred to as Gross Domestic Income(GDI).
\begin{equation}
GDP = W + I + R + \Pi
\end{equation}

$W$ - Wage, income earned by workers

$I$ - Interest, income earned by investors, or lenders

$R$ - Rent, income earned by landlords

$\Pi$ - Profit, income earned by entrepreneurs

\section{The Expenditure Method}
This method measures the expenditures spent in the economy, this method is the most common you may see on the textbooks.
\begin{equation}
GDP = C + I + G + NX
    = C + I + G + (X-M)
\end{equation}

$C$ - Consumption, spending of households

$I$ - Investment, spending of investors

$G$ - Government expenditure, spending of government

$NX$ - Net Exports, spending of overseas

$X$ - Exports

$M$ - Imports, it needs to be substracted since it is not produced domestically, refer to the definition of GDP, it should not be included.


\end{document}